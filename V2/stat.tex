%Explanation of the likelihood methods used, short explaination of the likelihood parts and constraints for combination, joint usage of nuisance parameters {Leff}.

\subsection{Statistical Inference }
\label{sec:LikelihoodFunction}
The statistical interpretation of data is performed by means of a binned profiled likelihood method, in which hypothesis testing relies upon a likelihood ratio test statistic, $\tilde{q}$, 
and its asymptotic distributions~\cite{asympt}. The two analysis channels are statistically independent once certain common nuisance parameters are factored out, so we combine them by multiplying their likelihoods together to produce a joint likelihood.
Both analyses parametrize the NR relative scintillation efficiency, $\Leff$, based on existing measurements~\cite{run8Result}. Its uncertainty is the major contributor to energy scale uncertainties and is considered as correlated between the two analysis channels via a separate nuisance likelihood term.
Throughout this study, all the parameters related to systematic uncertainties are assumed to be normally distributed.
%for the low ($\Llike_{\mathrm{lowE}}$) and high  ($\Llike_{\mathrm{highE}}$) energy channels as shown in equation~\ref{eq:FullLikelihood}. 
%\begin{equation}
%\label{eq:FullLikelihood}
%\Llike = \Llike_{\mathrm{lowE}} \times \Llike_{\mathrm{highE}} \times \Llike_{\Leff}
%\end{equation}

For the low energy channel an extended likelihood function is employed which is very similar to the one reported in~\cite{Aprile:2011hx} and described in detail in~\cite{xe100_run_combination}. 
The ROI discriminating (y,cS1)-plane is divided into eight WIMP mass dependent bands where events are counted. This binned approach is extended with the corresponding cS1-projected pdf of each band. The total normalization of the background is fit to data, and an uncertainty is assigned to the relative normalization of each band according to the corresponding statistical uncertainty of the calibration sample.
%The MLE of the background expectation in each band is constrained by the statistical uncertainty of the calibration sample in that band. 
Signal shape variations due to energy scale uncertainty are modeled via simulation. Additionally to $\Leff$, uncertainties due to the charge yield function $\Qy$ are parametrized as described in~\cite{DataMCXenon}.

The high energy channel analysis employs a binned likelihood function. Observed and expected event yield are compared in the nine ROI (y,cS1)-bins described in section~\ref{subsubsec:HighE}. 
Given the large statistical uncertainty of the background model the above extended likelihood approach is not repeated here.
Instead, the MLE of the background expectation in each bin is constrained by the statistical uncertainty of the calibration sample, while the total 
normalization is fit to the data. Additionally, to account for potential mismodeling of the expected background distribution, mainly due to anomalous multiple scatter events,
a systematic uncertainty of 20\% is assigned independently to each bin. In this channel, uncertainty on the signal acceptance of analysis selections are computed for each signal hypothesis using the parametrized acceptance curve shown in Figure~\ref{fig:Acc}.
Uncertainties on the signal model (y,cS1) distribution due to $^{241}$AmBe sample statistical fluctuations, as well as energy scale shape variation due to $\Leff$ uncertainties, are implemented.





%For the high energy channel a binned likelihood fuction has been chosen, defined in Eq.~\ref{eq:HighELikelihood}
%\begin{equation}
%\label{eq:HighELikelihood}
%\Llike_{\mathrm{highE}}(c_k^2,\Leff) = \prod_{i} \big( Poiss(n^{obs}_{i}~|~n^{s}_{i} + n^{b}_{i}) \, \times Gauss(\eta^{b}_{i}) \big) ~ \times \Llike_{stat}(\epsilon^{s}_{j},\epsilon^{b}_{i}) \times \Llike^s_{unc}(\Leff, A)
%\end{equation}
%where the product goes over all 9 bins, $\epsilon^{b}_{i}$ is the fraction of background event in \textbf{bin} $i$ and  $\epsilon^{s}_{j}$ is the fraction of AmBe data in \textbf{band} $j$. This means the uncertainty on the signal is assessed per band. $n_i^s = N_{tot}^s(c_k^2,\Leff) \times z_{i,j}^s(\Leff ,\epsilon^{s}_{j})$ is the number of signal events in bin i, $z_{i,j}^s(\Leff, \epsilon^{s}_{j})$ is the fraction of signal events in bin i which is in band j. $n_i^b = N_{tot}^{cal} \times \tau \times \epsilon^{b}_{i}(\eta^{b}_{i})$ is the number of background events in bin i. $\tau$ is the overall normalization of background to data, and is a free parameter.  
%
%The statistical uncertainties on the bins are constrained via:
%\begin{equation}
%\label{eq:LstsatHighE}
%\Llike_{stat}(\epsilon^{s}_{j},\epsilon^{b}_{i}) = \prod_{i}Poiss(N_i^{cal}~|~N_{tot}^{cal} \times \epsilon^{b}_{i}) \times \prod_j Poisson(N_j^{AmBe} | N_{tot}^{AmBe} \times \epsilon^{s}_{j})
%\end{equation}
%while the uncertainties on the signal model are treated in 
%\begin{equation}
%\Llike^s_{unc}(\Leff, A) = Gauss(A) \times Gauss(\Leff)
%\end{equation}
%
%The uncertainty on the acceptance depend on the expected signal and varies between $\sim$0.01\% to $\sim$7\%. $\eta^{b}_{i}$ is the background systematic uncertainty in bin i and is taken to be 20\%. An additional shape uncertainty due to $\Leff$ is also considered.   
%
%The lowE likelihood function is adopted from previous analyses and is \begin{equation}
%\label{eq:LowELikelihood}
%\Llike_{lowE} = \Llike_1(c_k^2,\Leff,\Qy) \Llike_2(\epsilon_b) \Llike_3(\Leff,\Qy) .
%\end{equation}
%
%\begin{equation} \label{eq:LowELikelihood1}
%\begin{split}
%\Llike_1(c_k^2,\Leff,\Qy) = &\prod_j Poiss(n^j|\epsilon_s^jM_s(c_k^2)+\epsilon_b^jM_b) \times  \\
%&\prod_{i=1}^{n^{i,j}} \frac{\epsilon_s^j M_s(c_k^2)f_s^j(cS1^i) + \epsilon_b^j M_bf_b^j(cS1^i)}{\epsilon^j_sM_s + \epsilon^j_bM_b} ,
%\end{split}
%\end{equation}
%
%where $f^j_s$ and $f^j_b$ are the probability density functions of the signal and background respectively in band j. and $M_s$ and $M_b$ are the maximum likelihood estimators for the total number of signal and background events respectively.
%\begin{equation}
%\Llike_2 = \prod_j Poiss(n^j_b | \epsilon_b^jN_b)
%\end{equation} 
%
%
