
\subsection{Low Energy}
\label{subsec:LowE}
This analysis channel relies on the re-analysis of run~II data described in~\cite{xe100_run_combination}. The region of interest, the background 
expectation models, data selections and their acceptances are mostly unchanged and so are only briefly summarized here. Differences with respect to said results are highlighted when present.

The region of interest for this channel spans the ($y,cS1$)-plane and is shown in Figure~\ref{fig:phasespace}.  The lower 
bound on $y$ corresponds to a 3\,$\sigma$ acceptance quantile (as a function of cS1) of a 20\,GeV WIMP mass signal model, while the upper bound is fixed at y\,=\,2.7.
The range in cS1 is selected as 3 to 30\,PE. This ROI is further divided into WIMP mass dependent sub-regions (also called bands) arranged in a way to achieve constant expected signal density in each region, as described in~\cite{xe100_run_combination}.

Other than falling into the ROI, an event should fulfill several additional selection criteria such as data quality and noise cuts,
an event veto in the presence of energy release in the outer LXe shield, energy selection and S2 threshold cuts,
selection of single-scatter events, and a predefined fiducial volume of 34\,kg. More details on these selection criteria and their 
relative WIMP signals acceptances can be found in~\cite{Aprile:2012vw,xe100_run_combination}. 

In addition, this analysis channel uses the post unblinding cuts and data reprocessing described in~\cite{xe100_run_combination}. 
%%%%% MAYBE THIS CAN BE COMMENTED OUT %%%%%%
To summarise the main features, data is reprocessed with an improved (S1,S2) classification algorithm, and a new cut targeted to suppress data periods with non-random occurrence of lone-S1 (an S1 without 
any correlated S2) events is applied. 
%%%%%%%%%%%%%%%%%%%%%%%%%%%%%%%%%%%%%%%%%%%%%
Finally, this channel does not employ a variable lower S1 threshold as a function of the event position, but a fixed 
lower threshold cut on cS1 at 3\,PE, converse to what was reported in~\cite{xe100_run_combination}.

The expected background is modeled separately for ER and NR contributions which are then scaled to exposure and added together.
The NR background is estimated by Monte Carlo simulation and accounts for the radiogenic and cosmogenic neutron
contributions~\cite{Aprile:2013tov}.
%neutrons from ambient materials and neutrons induced by cosmic ray showers. 
The ER background is parametrized as the linear combination of Gaussian-shaped and non-Gaussian components.
The first is obtained via a parametric fit of the $^{60}$Co and $^{232}$Th calibration data, as discussed in~\cite{xe100_run10_si}.
In constrast, the expected distribution and yields for the non-Gaussian population, consisting of anomalous events such as those 
presenting incomplete charge collection or accidental coincidence of uncorrelated S1s and S2s,  
are evaluated via dedicated techniques described in~\cite{xe100_run_combination}.

Systematic uncertainties on the background model arising from the Gaussian parametrized fit plus the normalisations of the NR and non-Gaussian components have been evaluated. It has been shown that the propagated errors are conservatively within the calibration samples (used to assess the model) statistical uncertainties, 
thus  chosen as the overall uncertainty of the model~\cite{xe100_run_combination}. 
