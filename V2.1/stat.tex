%Explanation of the likelihood methods used, short explaination of the likelihood parts and constraints for combination, joint usage of nuisance parameters {Leff}.

\subsection{Statistical Inference}
\label{sec:LikelihoodFunction}
The statistical interpretation of data is performed using a binned profile likelihood method, in which hypothesis testing relies upon a likelihood ratio test statistic, $\tilde{q}$, 
and its asymptotic distributions~\cite{asympt}. The two analysis channels are combined by multiplying their likelihoods together to produce a joint likelihood. 
Both analyses parametrize the NR relative scintillation efficiency, $\Leff$, based on existing measurements~\cite{run8Result}. Its uncertainty is the major contributor to energy scale uncertainties and is considered as correlated between the two analysis channels via a joint nuisance likelihood term.
Throughout this study, all the parameters related to systematic uncertainties are assumed to be normally distributed.
%for the low ($\Llike_{\mathrm{lowE}}$) and high  ($\Llike_{\mathrm{highE}}$) energy channels as shown in equation~\ref{eq:FullLikelihood}. 
%\begin{equation}
%\label{eq:FullLikelihood}
%\Llike = \Llike_{\mathrm{lowE}} \times \Llike_{\mathrm{highE}} \times \Llike_{\Leff}
%\end{equation}

For the low energy channel an extended likelihood function is employed which is very similar to the one reported in~\cite{Aprile:2011hx} and described in detail in~\cite{xe100_run_combination}. 
The \sout{ROI discriminating} (y,cS1)-plane is divided into eight WIMP mass dependent bands where events are counted. This binned approach is extended with the corresponding cS1-projected PDF of each band. The total normalization of the background is fit to data, and an uncertainty is assigned to the relative normalization of each band according to the corresponding statistical uncertainty of the calibration sample.
%The MLE of the background expectation in each band is constrained by the statistical uncertainty of the calibration sample in that band. 
\ale{Signal shape variations due to energy scale uncertainty are modeled via simulation. These include  the said $\Leff$ uncertainties and additionally 
the charge yield uncertainties, which are parametrized based on $\Qy$ measurement as described in~\cite{DataMCXenon}.}

The high energy channel analysis employs a binned likelihood function. Observed and expected event yield are compared in the nine ROI (y,cS1)-bins described in section~\ref{subsubsec:HighE}. 
Given the large statistical uncertainty of the background model the above extended likelihood approach is not repeated here.
Instead, the maximum likelihood estimation of the background expectation in each bin is constrained by the statistical uncertainty of the calibration sample, while the total 
normalization is fit to the data. Additionally, to account for potential mismodeling of the expected background distribution, mainly due to anomalous multiple scatter events,
a systematic uncertainty of 20\% is assigned independently to each bin. In the high energy channel, uncertainty on the signal acceptance of analysis selections are computed for each signal hypothesis using the parametrized acceptance curve shown in Figure~\ref{fig:Acc}.
Uncertainties on the signal model (y,cS1) distribution due to $^{241}$AmBe sample statistical fluctuations, as well as energy scale shape variation due to $\Leff$ uncertainties, are taken into account.



