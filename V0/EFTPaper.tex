%%%%%%%%%%%%%%%
% EFT Paper
% v.1
% R. Itay & B. Farmer & A. Manfredini
%%%%%%%%%%%%%%%

\RequirePackage{lineno}
\documentclass[twocolumn, showpacs, showkeys, amsmath, amssymb, floatfix]{revtex4} 
\usepackage[colorlinks=true, citecolor=green, filecolor=blue, linkcolor=blue, urlcolor=blue]{hyperref}
\usepackage{color}
\usepackage{graphicx}
\usepackage[perpage]{footmisc}
	
\newcommand \Leff{$\mathcal{L}_{\mathrm{eff}}$}
\newcommand \Ly{$L_{\mathrm{y}}$}
\newcommand \Qy{$\mathcal{Q}_{\mathrm{y}}$}
\newcommand \keVcc{$\mathrm{keV}/\mathrm{c}^2$}
\newcommand \keVr{$\mathrm{keV_{nr}}$}
\newcommand \keVee{$\mathrm{keV_{ee}}$}
\newcommand{\Xehund}{{XENON100}} 
\newcommand{\Xeten}{{XENON10}}
\newcommand{\Xe}{{\sc Xe}}
\newcommand{\n}[1]{\mathrm{#1}}

\newcommand{\cc}[1]{$c_{#1}^2\timesm_{Weak}^4$}


%%%%%%%%%

\begin{document}
\linenumbers 

\title{Effective Field Theory Approach to Elastic Scattering of Dark Matter in  \Xehund\ Detector 225 live days run}
%\input{AuthorList}

%\date{\today}

\begin{abstract} 

bla bla bla

\end{abstract}

\pacs{}
\keywords{Dark Matter, EFT, Xenon}

\maketitle 
\section{Introduction}
\section{The \Xehund\  Detector}
The \Xehund\ detector is a cylindrical (30cm height X 30cm diameter) dual phase Xenon Time Projection Chamber (TPC) that holds 62 kg of Liquid \Xe\ (LXe) targets ~\cite{xe100_instr2012}. It operates at the Laboratori Nazionali del Gran Sasso (LNGS) in Italy. The detector consists a total of 242 1”-square Hamamatsu R8520-AL photomultiplier tubes (PMTs) employed in two arrays, at the top part (in the gas phase) and in the bottom immersed in LXe. a Particle interacting with the LXe deposits energy that creates both excited and ionized states. De-excitation creates a prompt scintillation signal ($S1$). ,  Ionized electrons are drifted in an electric field of $530$V/cm towards the liquid-gas interface, where they are extracted via a larger electric field of $\sim12$kV/cm. These electrons generates a proportional scintillation, which is called $S2$. The spatial distribution of the $S2$ signal on the top PMT array, determines the X-Y position, while the time difference between the two signals gives the z-coordinate, and thus a 3D position reconstructions is achieved.

The ratio of S2/S1 is different weather the interaction is nuclear recoil (NR) or electronic recoil (ER) and thus this ratio is used as a discriminator between ER background coming from $\gamma$, $\beta$ and NR signal coming from a WIMP. 

In previous \Xehund\ analyses the determination of the recoil energy was based on the size of S1 and the scintillation efficiency for the nuclear recoils, \Leff ~\cite{xe100_run10_si}. However in the last analysis ~\cite{xe100_run_combination} a new method was adopted taking into advantage also the S2 signal.
      

\section{The Analysis}
\section{Results}




%%% BIBLIOGRAPHY %%%

\bibliographystyle{apsrev}
\bibliography{EFTPaperBib}

\end{document}
