
\subsection{Low Energy}
\label{subsec:LowE}
This analysis channel take advantage of the re-analysis of run~II data described in~\cite{xe100_run_combination}, the region of interest, the background 
expectation models, data selections and their acceptances are mostly unchanged with respect to said result and only briefly summarized here, 
differences are highligted when present. 

The region of interest for this channel in the (y,cS1)-plane is shown in Figure~\ref{fig:phasespace},  the lower and upper bound on y are the 
3\,$\sigma$ acceptance quantile (as a function of cS1) of a 20\,GeV WIMP mass signal model and y\,=\,2.7, respectively, while cS1 is 
constrained to be within 3 and 30\,PE. The ROI is further divided into WIMP mass dependent sub-regions (also called bands) arranged in a way 
to achieve constant signal density in each region, as described in~\cite{xe100_run_combination}.

Other than falling into the ROI an event should fullfill several additional selection criteria such as, data quality and noise cuts,
an event veto in presence of energy release in the outer LXe shield, energy selection and S2 threshold cut, 
selection of single-scatter event and a predefined fiducial volume of 34\,kg. More details on these selection criteria and on their 
relative acceptances on WIMP signals can be found in~\cite{Aprile:2012vw,xe100_run_combination}. 
Furthermore, this analysis channel uses the post unblinding cuts described in~\cite{xe100_run_combination}, 
%%%%% MAYBE THIS CAN BE COMMENTED OUT %%%%%%
these additional selections consist of a cut targeted to suppress data periods with non random occurrence of lone-S1 (an S1 without 
any correlated S2) and of a data reprocessing with an improved (S1,S2) classification algorithm.
%%%%%%%%%%%%%%%%%%%%%%%%%%%%%%%%%%%%%%%%%%%%%
Finally, this analysis does not employ a variable lower S1 threshold as a function of the light correction efficiency, but a fixed 
lower threshold cut on cS1 at 3\,PE, conversely to what reported in~\cite{xe100_run_combination}.

The expected background is modeled separately for ER and NR contributions which are then scaled to exposure and added together.
The NR background is estimated by Monte Carlo simulation and it accounts for the radiogenic and cosmogenic neutrons
contributions~\cite{Aprile:2013tov}.
%neutrons from ambient materials and neutrons induced by cosmic ray showers. 
The ER background is parametrized as the superimposion of a Gaussian-shaped and of a non-Gaussian components.
The first one is obtained via  parametric fit of the $^{60}$Co and $^{232}$Th calibration data, as discussed in~\cite{xe100_run10_si}.
While the the expected distribution and yields for the non-Gaussian population, consisting of anomalous events such as those 
presenting incoplete charge collection or accidental coincidence of uncorrelated S1s and S2s,  
are evaluated employing dedicated techniques described in~\cite{xe100_run_combination}.
%Or maybe simply the ER bkg take into account this and that as described in~\cite{}.

\textcolor{red}{Need to define benchmark to give expected and observed yield in a table for both channels.}

\textcolor{red}{Uncertainties.}
