\subsection{The Likelihood Function}
%Explanation of the likelihood methods used, short explaination of the likelihood parts and constraints for combination, joint usage of nuisance parameters {Leff}.
\subsection{The Likelihood Function}
\label{sec:LikelihoodFunction}
We use a binned profile likelihood approach to constrain the coupling constants $c_k$ for each operator $\mathcal{O}_k$. The full likelihood function is constructed as a product of the likelihoods of each energy region 
\begin{equation}
\label{eq:FullLikelihood}
\Llike = \Llike_{\mathrm{lowE}(c_i^2,\Qy,\Leff)} \times \Llike_{\mathrm{highE}(c_i^2,\Leff)}  .
\end{equation}

The highE likelihood function is defined in Eq.~\ref{eq:HighELikelihood}
\begin{equation}
\label{eq:HighELikelihood}
\Llike_{\mathrm{highE}}(c_k^2,\Leff) = \prod_{i} \big( Poiss(n^{obs}_{i}~|~n^{s}_{i} + n^{b}_{i}) \, \times Gauss(\eta^{b}_{i}) \big) ~ \times \Llike_{stat}(\epsilon^{s}_{j},\epsilon^{b}_{i}) \times \Llike^s_{unc}(\Leff, A)
\end{equation}
where the product goes over all 9 bins, $\epsilon^{b}_{i}$ is the fraction of background event in \textbf{bin} $i$ and  $\epsilon^{s}_{j}$ is the fraction of AmBe data in \textbf{band} $j$. This means the uncertainty on the signal is assessed per band. $n_i^s = N_{tot}^s(c_k^2,\Leff) \times z_{i,j}^s(\Leff ,\epsilon^{s}_{j})$ is the number of signal events in bin i, $z_{i,j}^s(\Leff, \epsilon^{s}_{j})$ is the fraction of signal events in bin i which is in band j. $n_i^b = N_{tot}^{cal} \times \tau \times \epsilon^{b}_{i}(\eta^{b}_{i})$ is the number of background events in bin i. $\tau$ is the overall normalization of background to data, and is a free parameter.  

The statistical uncertainties on the bins are constrained via:
\begin{equation}
\label{eq:LstsatHighE}
\Llike_{stat}(\epsilon^{s}_{j},\epsilon^{b}_{i}) = \prod_{i}Poiss(N_i^{cal}~|~N_{tot}^{cal} \times \epsilon^{b}_{i}) \times \prod_j Poisson(N_j^{AmBe} | N_{tot}^{AmBe} \times \epsilon^{s}_{j})
\end{equation}
while the uncertainties on the signal model are treated in 
\begin{equation}
\Llike^s_{unc}(\Leff, A) = Gauss(A) \times Gauss(\Leff)
\end{equation}

The uncertainty on the acceptance depend on the expected signal and varies between $\sim$0.01\% to $\sim$7\%. $\eta^{b}_{i}$ is the background systematic uncertainty in bin i and is taken to be 20\%. An additional shape uncertainty due to $\Leff$ is also considered.   

The lowE likelihood function is adopted from previous analyses and is \begin{equation}
\label{eq:LowELikelihood}
\Llike_{lowE} = \Llike_1(c_k^2,\Leff,\Qy) \Llike_2(\epsilon_b) \Llike_3(\Leff,\Qy) .
\end{equation}

\begin{equation} \label{eq:LowELikelihood1}
\begin{split}
\Llike_1(c_k^2,\Leff,\Qy) = &\prod_j Poiss(n^j|\epsilon_s^jM_s(c_k^2)+\epsilon_b^jM_b) \times  \\
&\prod_{i=1}^{n^{i,j}} \frac{\epsilon_s^j M_s(c_k^2)f_s^j(cS1^i) + \epsilon_b^j M_bf_b^j(cS1^i)}{\epsilon^j_sM_s + \epsilon^j_bM_b} ,
\end{split}
\end{equation}

where $f^j_s$ and $f^j_b$ are the probability density functions of the signal and background respectively in band j. and $M_s$ and $M_b$ are the maximum likelihood estimators for the total number of signal and background events respectively.
\begin{equation}
\Llike_2 = \prod_j Poiss(n^j_b | \epsilon_b^jN_b)
\end{equation} 

The last part of the likelihood treats the constraints on the nuisance parameter $\Qy$ and $\Leff$ normally constraining them. The latter is a joined nuisance parameter for both energy regions. 


Explanation of the likelihood methods used, short explaination of the likelihood parts and constraints for combination, joint usage of nuisance parameters {Leff}.
%\subsubsection{High Energy}
%likelihood function + uncertainties 
%\subsubsection{Low Energy}
%ref to run combination.




